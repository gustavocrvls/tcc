% LISTA DE ABREVIATURAS E SIGLAS----------------------------------------------------------

\begin{siglas} 
    \item [AC] Atividade Complementar
    \item [ACs] Atividades Complementares
    \item [ACC] Atividade Curricular Complementar
    \item [ACCs] Atividades Curriculares Complementares
    \item [ACGs] Atividades Curricularesde Graduação
    \item [CAAd] Coordenadoria de Acompanhamento Acadêmico
    \item [FACEEL] Faculdade de Computação e Engenharia Elétrica
    \item [HTML] \textit{HyperText Markup Language}
    \item [IFRS] Instituto  Federal  do  Rio  Grande  do  Sul
    \item [IGE] Instituto de Geociências e Engenharias
    \item [JWT] \textit{JSON Web Token}
    \item [MEC] Ministério da Educação
    \item [ORM] \textit{Object-relational mapping}
    \item [SEO] \textit{Search Engine Optimization}
    \item [SQL] \textit{Standard Query Language}
    \item [REST] \textit{REpresentational State Transfer}
    \item [RF] Requisito Funcional
    \item [RNF] Requisito Não Funcional
    \item [SAAMS] Sistema  de  Assistência  Estudantil  e  Acompanhamento  Acadêmico)
    \item [SIGAA] Sistema Integrado de Gestão de Atividades Acadêmicas
    \item [SQL] \textit{Structured Query Language}
    \item [UFF] Universidade Federal Fluminense
    \item [UFRN] Universidade Federal do RioGrande  do  Norte
    \item [UML] \textit{Unified Modeling Language}
    \item [UNIFESSPA] Universidade Federal do Sul e Sudeste do Pará
    
\end{siglas}

% OBSERVAÇÕES-----------------------------------------------------------------------------
% Altere a lista acima para definir os acrônimos e siglas utilizados neste trabalho
