% RESUMO--------------------------------------------------------------------------------

\begin{resumo}[RESUMO]
\begin{SingleSpacing}

As Atividades Curriculares Complementares (ACCs) são componentes obrigatórios de qualquer curso de graduação. A Secretaria da Educação Superior do Ministério da Educação caracteriza as atividades complementares como tendo a finalidade de enriquecer o processo de ensino-aprendizagem, bem como o desenvolvimento profissional e social através da interação dos discentes com a prática associada àquilo que aprendem em sala de aula. A gestão dessas atividades gera um desafio enorme às faculdades, que recorrem a estratégias tradicionais como o envio dos certificados por correio eletrônico por parte do discente, ou mesmo a entrega presencial destes. A Faculdade de Computação e Engenharia Elétrica (FACEEL) também sofre o desafio da gestão dessas atividades, exigindo que os discentes enviem os certificados nos últimos períodos do curso, que é quando a matéria de atividades complementares é ofertada. Os problemas gerados por isso são o gargalo durante a avaliação das atividades, pois a coordenação da faculdade terá que fazer a validação e integralização de todos os discentes no mesmo período; há também a falta de acompanhamento dos discentes quanto à pontuação exata que eles possuem, o que causa o receio de se chegar nos semestres finais do curso e não haver carga horária suficiente para integralização da matéria. Diante dos desafios citados, propôs-se o projeto e desenvolvimento de uma ferramenta web para gestão automatizada de atividades complementares dos discentes dos cursos da FACEEL, denominado KeeMe, que é uma abreviação de \textit{"Keep it to Me"} ou "Guarde isso para mim". O trabalho envolve as etapas de coleta e análise de requisitos necessários para o sistema, passando pela parte de modelagem até a codificação da ferramenta propriamente dita. Serão mostradas as estratégias utilizadas para o desenvolvimento da ferramenta, tecnologias aplicadas, além dos resultados da avaliação de qualidade do sistema proposto.

\end{SingleSpacing}

\textbf{Palavras-chave}: Atividades Complementares; Gestor de Informação; Engenharia de Software; Sistema Web;

\end{resumo}

% OBSERVAÇÕES---------------------------------------------------------------------------
% Altere o texto inserindo o Resumo do seu trabalho.
% Escolha de 3 a 5 palavras ou termos que descrevam bem o seu trabalho 

