% ABSTRACT--------------------------------------------------------------------------------

\begin{resumo}[ABSTRACT]
\begin{SingleSpacing}

Complementary Curricular Activities (ACCs) are mandatory components of any undergraduate course. The Higher Education Secretariat of the Ministry of Education characterizes complementary activities as having the purpose of enriching the teaching-learning process, as well as professional and social development through the interaction of students with the practice associated with what they learn in the classroom. The management of these activities creates an enormous challenge for the faculties, which resort to traditional strategies such as the sending of certificates by e-mail by the student, or even their face-to-face delivery. The Faculty of Computing and Electrical Engineering (FACEEL) also faces the challenge of managing these activities, requiring students to send certificates in the last periods of the course, which is when the subject of complementary activities is offered. The problems generated by this are the bottleneck during the evaluation of activities, since the coordination of the faculty will have to validate and pay all students in the same period; there is also a lack of monitoring by students as to the exact score they have, which causes the fear of arriving in the final semesters of the course and not having enough hours to complete the material. In view of the aforementioned challenges, it was proposed to design and develop a web tool for automated management of complementary activities for students of FACEEL courses, called KeeMe, which is an abbreviation of "Keep it to Me". The work involves the steps of collecting and analyzing requirements necessary for the system, going from the modeling part to the coding of the tool itself. The strategies used for the development of the tool, applied technologies, as well as the results of the quality assessment of the proposed system will be shown.


\end{SingleSpacing}

\textbf{Keywords}: Complementary activities; Information Manager; Software development; Web application;

\end{resumo}

% OBSERVAÇÕES---------------------------------------------------------------------------
% Altere o texto inserindo o Abstract do seu trabalho.
% Escolha de 3 a 5 palavras ou termos que descrevam bem o seu trabalho 
