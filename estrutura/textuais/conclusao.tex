% CONCLUSÃO--------------------------------------------------------------------

\chapter{CONSIDERAÇÕES FINAIS}
\label{chap:conclusao}

Neste trabalho foi desenvolvida a aplicação KeeMe, ferramenta criada com o objetivo de informatizar os processos de avaliação de Atividades Curriculares Complementares dos discentes da FACEEL. Tal ferramenta foi construída utilizando os princípios da engenharia de software, com foco na simplicidade e na qualidade.

O início do desenvolvimento se deu através da coleta de requisitos do sistema com as partes interessadas de forma a se levantar as reais necessidades que a ferramenta deveria suprir. Tais requisitos foram coletados através de reuniões, e da leitura da Resolução de ACC presente no Anexo \ref{anexo:novaResolucaoDeACC}; Após isso, foi feita a modelagem dos requisitos, transformando aquilo que foi coletado em representações visuais que mostrassem quais os ``atores" (tipos de usuário) envolvidos no sistema, quais as entidades que deveriam compor a aplicação, como essas entidades relacionam-se entre si, e quais os fluxos que os usuários executarão dentro do sistema. Nesta etapa também foram desenvolvidos protótipos da interface do sistema com o objetivo de se ter uma visão mais clara dos dados a serem exibidos e das funcionalidades que irão compor o sistema final. Após isso, deu-se a etapa de codificação da aplicação, nesta etapa buscou-se seguir a estrutura criada através dos diagramas construídos na etapa de modelagem, além das interfaces criadas pela prototipagem.

Após o desenvolvimento da aplicação, houve a necessidade de se fazer a avaliação de sua qualidade, tal avaliação teve como objetivo obter uma visão clara do quão bem construída estava a aplicação. Os resultados demonstram que a aplicação possui qualidade nos critérios avaliados através da ferramenta Lighthouse, que segundo \cite{lighthouse} se trata de uma ferramenta de testes automatizados de qualidade em páginas web.

Através do uso da aplicação KeeMe, os discentes poderão ter um acompanhamento mais transparente e simplificado quanto à sua obtenção de pontos em ACCs, assim como um ambiente onde possam fazer o armazenamento dos certificados de suas atividades. Já a coordenação será contemplada com um processo simplificado de avaliação de ACCs e das pontuações obtidas pelos alunos.

Como trabalhos futuros, propõe-se a integração da ferramenta KeeMe com o Sistema Integrado de Gestão de Atividades Acadêmicas (SIGAA) da Universidade Federal do Sul e Sudeste do Pará (UNIFESSPA); a criação de um módulo de geração de relatórios das ACCs enviadas e das pontuações obtidas pelos discentes; e ainda a criação de um módulo que contemple as atividades de extensão realizada pelos discentes.
