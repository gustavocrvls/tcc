% INTRODUÇÃO---------------------------------------------------------------
\chapter{INTRODUÇÃO}
\label{chap:introducao}

Segundo o \cite{min_educ_perguntas} as Atividades Complementares “têm a finalidade de enriquecer o processo de ensino-aprendizagem, privilegiando a complementação da formação social e profissional”. Tais atividades são obrigatórias dentro de qualquer curso e tem como objetivo fazer uma associação entre a teoria aprendida dentro da sala de aula e a prática. Tais atividades podem ser divididas em três categorias, sendo estas Ensino, Pesquisa e Extensão.

As Atividades Complementares dentro da Faculdade de Computação e Engenharia Elétrica (FACEEL) são denominadas Atividades Curriculares Complementares e são regidas pela Resolução FACEEL - IGE 002/2014 desenvolvida por \cite{faceel2014regulamento} que faz a medição das atividades utilizando um sistema de pontos, fixando em 61 (sessenta e um) o mínimo necessário para integralização da matéria. A matéria deverá ser realizada no 7º semestre no curso de Sistemas de Informação e no 10º nos cursos de Engenharia Elétrica e Engenharia da Computação.

O controle dessas ACCs é um desafio tanto para a coordenação da FACEEL quanto para os discentes. A coordenação, ao final dos semestres precisa receber os certificados de ACC de todos os alunos, fazer a validação desses certificados e por último fazer o cálculo da pontuação gerada pelas ACCs. Já os discentes acabam tendo que guardar todos os certificados obtidos por eles no decorrer do curso, além de fazerem um cálculo por si próprios para analisar se já possuem a pontuação mínima, o maior problema gerado aos discentes é a incerteza do total de pontos que ele possui.

Diante dos desafios citados, foi necessária a criação de uma ferramenta que informatizasse esse processo e o torna-se simplificado. A ferramenta proposta deveria ser capaz de armazenar os certificados enviados pelos discentes, fazer a contagem dos seus pontos, e permitir aos coordenadores fazer a validação da pontuação dos discentes. Esse trabalho abordará a construção dessa solução, iniciando pela fase da coleta dos requisitos com as partes interessadas, após isso será mostrado as estratégias de desenvolvimento do projeto e artefatos criados durante a modelagem e prototipação da solução, e detalhes sobre a codificação em si da ferramenta, além das métricas utilizadas para se avaliar a sua eficiência.

\section{Motivação}

A motivação deste trabalho se encontra na resolução de uma problemática relacionada à gestão de Atividades Curriculares Complementares nos cursos da na Faculdade de Computação e Engenharia Elétrica (FACEEL). Os discentes devem reunir os certificados das Atividades Curriculares Complementares (ACCs) até os últimos semestres dos cursos, fazendo por si só o controle dos pontos obtidos. Tal cenário gera um desafio à coordenação que terá que avaliar e integralizar os certificados gerados por vários discentes no mesmo período, gerando um gargalo nessa avaliação. Os discentes também são afetados pois, por mais que possuam o controle de pontos obtidos, há o receio de que se chegar ao último semestre sem possuir a quantidade exigida de pontos. Tais desafios motivaram a criação de uma solucação capaz de informatizar o processo de obtenção das ACCs e torná-lo simples e transparente aos discentes e à coordenação.

\section{Objetivo Geral}

O objetivo geral deste trabalho é a coleta de requisitos, a modelagem e o desenvolvimento de um sistema web que seja capaz de informatizar o gerenciamento das ACCs (Atividades Curriculares Complementares) dos discentes, permitindo que estes tenham um acompanhamento transparente de suas pontuações. Em conjunto a isso, a solução deverá permitir a avaliação das ACCs por parte dos coordenadores dos cursos da FACEEL.

\section{Objetivos Específicos}

\begin{itemize}
    \item Coletar os requisitos funcionais e não-funcionais;
    \item Definir a estratégia para se implementar as atuais regras de ACC da FACEEL;
    \item Realizar a modelagem e desenvolvimento do sistema proposto; 
    \item Avaliar a qualidade da ferramenta proposta utilizando métricas voltadas a sistemas web.
    
\end{itemize}

\section{Estrutura do Trabalho}

Este trabalho está estruturado em 7 capítulos que estão organizados da seguinte forma:

\begin{description}
    \item[\textbf{Capítulo II}] Apresenta uma revisão literária acerca da engenharia de software, e da gestão da informação, tópicos importantes dentro do contexto da proposta criada.
    \item[\textbf{Capítulo III}] Mostra alguns trabalhos relacionados com a proposta da monografia, destacando suas principais contribuições e temáticas abordadas.
    \item[\textbf{Capítulo IV}] Descreve a metodologia utilizada, explanando sobre o ambiente de desenvolvimento e as ferramentas utilizadas.
    \item[\textbf{Capítulo V}] Discorre sobre a proposta criada, fazendo pontuações sobre a forma como o sistema está estruturado, e sobre a forma que as informações interagem na aplicação.
    \item[\textbf{Capítulo VI}] Descreve os resultados encontrados na avaliação da ferramenta proposta.
    \item[\textbf{Capítulo VII}] Apresenta as considerações finais da monografia.
\end{description}
