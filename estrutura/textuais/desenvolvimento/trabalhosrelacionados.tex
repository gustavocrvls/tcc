\chapter{TRABALHOS RELACIONADOS}
\label{chap:trabalhosrelacionados}

Existem na literatura vários trabalhos que focam no desenvolvimento de sistemas para a gestão da informação e de atividades curriculares complementares.
A seção \ref{sec:gestacademic} abordará trabalhos em gestores de informação voltados ao ambiente de aprendizado. Enquanto a seção \ref{sec:gestativcomplem} mostrará gestores de Atividades Curriculares Complementares, e que possuem o mesmo foco da ferramenta proposta.

\section{Gestores de Informação}
\label{sec:gestacademic}

\cite{cardoso2018desenvolvimento} aborda em seu trabalho as etapas da criação da ferramenta SAAMS (Sistema de Assistência Estudantil e Acompanhamento Acadêmico), desenvolvida mediante uma necessidade da Coordenadoria de Acompanhamento Acadêmico (CAAd) do Instituto
Federal do Rio Grande do Sul (IFRS). Conforme descrito por \cite{cardoso2018desenvolvimento}, todos os processos da CAAd eram realizados através de planilhas, ou até mesmo com arquivos de texto. Tal abordagem não era ideal, pois acabava gerando operações repetitivas e informações conflitantes ou inconsistentes. Diante disso, foi necessária a criação de uma ferramenta que auxiliasse nos processos da CAAd. \cite{cardoso2018desenvolvimento} descreve então as etapas do processo necessárias para a criação da ferramenta, desde a etapa de levantamento de requisitos até a codificação em si da aplicação.

Em \cite{barroca2013sigaa} é mostrado o processo de implantação de uma versão voltada para dispositivos móveis do Sistema Integrado
de Gestão de Atividades Acadêmicas (SIGAA) na Universidade Federal do Rio Grande do Norte (UFRN). Segundo \cite{barroca2013sigaa} havia uma necessidade de se atender à demanda crescente de dispositivos móveis com a criação de uma aplicação voltada a estes, porém esta aplicação deveria manter as funcionalidades já presentes no sistema, de forma que qualquer ação realizada pelo sistema em si pudesse ser realizada por um dispositivo móvel. O trabalho aborda as etapas que foram seguidas na criação da aplicação, desde a modelagem até a codificação em si. \cite{barroca2013sigaa} demonstra ainda uma avaliação dos resultados que mostraram um grande número de instalações em proporção ao número total de usuários do sistema.

Em seu trabalho \cite{gomes2012genome} descrevem a utilização da ferramenta GENOME que trabalha em consonância com o Moodle, um sistema livre de apoio ao aprendizado, para se fazer a avaliação da participação virtual e gerenciamento de notas dos discentes do Instituto Metrópole Digital da Universidade Federal do Rio Grande do Norte (UFRN). \cite{gomes2012genome} explica que a avaliação das notas dos alunos é feita levando em consideração tanto as atividades realizadas presencialmente,  que englobam presença, pontualidade,
resolução de exercícios, participação no encontro, seminários e questões desafios, quanto as realizadas à distância, como a participação de fóruns e chats. Todas essas avaliações são avaliadas e juntas compõem a nota final, o GENOME serve então como uma ferramenta que faz a padronização dessas avaliações, e cria relatórios mais completos do que os já existentes dentro da ferramenta Moodle.

% O trabalho de \cite{amin2018active} tem um viés mais técnico em relação aos sistemas abordados descreve a criação de uma abordagem de sistema de banco de dados para o desenvolvimento de uma aplicação de gestão acadêmica. Um Banco de Dados Ativo é descrito como capaz de capturar determinados eventos e realizar ações a partir disso. O sistema também faz o uso de regras, que são descritas como uma complementação de uma linguagem de programação utilizada no desenvolvimento de uma aplicação.

% No trabalho de \cite{sankar2018web} é descrito o desenvolvimento de uma aplicação para o gerenciamento acadêmico, visando automatizar as ações realizadas no ambiente acadêmico e assim eliminando os problemas gerados pela operação manual dessas ações. A aplicação foi desenvolvida em ASP e permite que os alunos possam ter aulas através dela.


\cite{rodriguez2018desenvolvimento} apresentam em seu trabalho a criação da ferramenta SolicitaUFF, um sistema acadêmico-administrativo com foco em usabilidade e acessibilidade. Segundo \cite{rodriguez2018desenvolvimento}, até então os processos desempenhados pela coordenação da Universidade Federal Fluminense (UFF) eram em grande parte feitos de forma manual, e os alunos precisavam comparecer à coordenação para realizar ações como: requerimentos de Atividades Complementares; dispensa em disciplinas; entre outras. A criação do sistema teve como foco aumentar a eficácia dos processos da UFF através da informatização dos processos, e assim diminuir o tempo perdido na execução manual dos mesmos.

\section{Gestores de Atividades Curriculares Complementares}
\label{sec:gestativcomplem}

\cite{de2018implementaccao} discorrem sobre o desafio das Instituições de Ensino Superior em fazer o controle das Atividades Complementares realizadas pelos discentes e ainda de tornar esse controle transparente. As Instituições utilizam diferentes estratégias, seja utilizando correio eletrônico, comparecimento presencial do discente ou um sistema específico, porém esse processo é desafiante no que tange a permitir que os discentes tenham um acompanhamento claro de seus avanços. \cite{de2018implementaccao} descrevem então sobre a criação da ferramenta \textit{Akademic} que seria uma ferramenta de Gestão de Atividades Complementares utilizável por qualquer Instituição.

\cite{silva2013processo} demonstram em seu trabalho a utilização da ferramenta Moodle, um ambiente virtual de aprendizado, para o gerenciamento das Atividades Complementares dos discentes da  Faculdade de Computação e Informática (FCI) da Universidade Presbiteriana Mackenzie (UPM). \cite{silva2013processo}. Através da utilização dessa ferramenta acarretou a diminuição da utilização de papel na FCI, além da criação de um canal onde alunos e professores poriam colaborar em discussões, e onde os alunos poderiam ter acesso às atividades da Faculdade.


\cite{cunha2014unipampa} explica que há um desafio no controle das Atividades Curriculares de Graduação (ACGs) na Universidade Federal do Pampa (UNIPAMPA), pois o processo de aproveitamento das ACGs é trabalhoso, exigindo que os discentes preencham a requisição desse aproveitamento, relacionando suas atividades com a tabela de ACGs da UNIPAMPA. O maior desafio está no fato de que os discentes muitas vezes esperam até o último período para fazer esse aproveitamento, gerando atrasos nessas requisições. Diante desse cenário \cite{cunha2014unipampa} descreve sobre a criação da ferramenta SmartACG que tem como objetivo a informatização dos processos de integralização de ACGs, dessa forma a ferramenta seria um automatizador de processos, que auxiliaria os alunos na integralização das ACGs.

Com o objetivo de relacionar as especificidades de cada ferramenta foi feita uma comparação entre eles, que pode ser encontrada na Tabela \ref{table:ComparacaoArtigos}, essa comparação também envolve o KeeMe, ferramenta desenvolvida neste trabalho. As seguintes características serão abordados na tabela:

\begin{itemize}
    \item \textbf{Plataformas:} Descreve quais as plataformas são abarcadas pelo sistema, ou seja, se ele é Web ou Móvel;
    \item \textbf{\textit{Status}:} Aborda a capacidade que o sistema tem de mostrar aos alunos se suas ACCs foram avaliadas, aprovadas ou reprovadas;
    \item \textbf{Personalização:} Descreve o tipo de personalização que cada tipo de ACC possuirá. Os campos avaliados são: Nome, que descreve o nome do tipo de ACC; Peso, relativo, por exemplo à quantidade de horas que são necessárias para que se consiga um ponto de ACC; e Limite, que é a pontuação máxima que um usuário poderá conseguir com determinado tipo de ACC.
\end{itemize}

\begin{table}[H]
\centering
\caption{Tabela comparativa dos trabalhos relacionados com a proposta.}
\label{table:ComparacaoArtigos}
\begin{tabular}{|C{3.64cm}|C{3.14cm}|C{2.78cm}|C{4.64cm}|}
\hline
\textbf{Ferramenta} & \textbf{Plataformas} & \textbf{\textit{Status}} & \textbf{Personalização} \\
\hline
Akademic & Web & Não  & Nome da ACC \\ \hline
Moodle & Web & N/A & Nome da ACC \\ \hline
SmartACG & Móvel & Sim & Nome da ACC \\ \hline
KeeMe & Web & Sim & Nome da ACC, Peso e Limite \\ \hline
\end{tabular}
\end{table}

A ferramenta Akademic de \cite{de2018implementaccao} é a mais próxima da solução proposta neste trabalho em questões de interface do usuário e usabilidade, permitindo que o usuário utilize o sistema tanto em um computador quanto em um celular sem limitações de funcionalidade, porém tem como limitações a utilização apenas de Horas para avaliar o avanço das ACCs, e da não conversão dessas horas em pontos, o que é algo essencial para a avaliação de ACCs na FACEEL. O trabalho de \cite{cunha2014unipampa}, possui a limitação de ser criado apenas para a plataforma Android, sem a possibilidade de acesso por um computador, e sem acesso pelos coordenadores. Já \cite{silva2013processo} apesar de também se tratar de uma ferramenta de informatização das ACCs, possui a limitação de ser baseado na plataforma Moodle, restringindo seu uso pelos discentes da FACEEL.


